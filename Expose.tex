
\documentclass[	DIV=calc, paper=a4,	fontsize=11pt, twocolumn]{scrartcl}

\usepackage[english]{babel}

\usepackage[utf8]{inputenc}
\usepackage[T1]{fontenc}
\usepackage{lmodern}

\renewcommand*{\familydefault}{\sfdefault}

\usepackage[pdftex]{graphicx}
\usepackage[svgnames]{xcolor}
\usepackage[hang, small,labelfont=bf,up,textfont=it,up]{caption}	% Custom captions under/above floats
\usepackage[protrusion=true,expansion=true]{microtype}				% Better typography

\title{Utilizing Modern Game Development Technologies for Neuro-Scientific Research}					% Title of your article 
\subtitle{Expose - Master Thesis}
\author{Markus Fleck : 337265}
\date{\today}

\begin{document} 

\maketitle

\begin{abstract}
% summarize at the end
\end{abstract}

\section*{Introduction}

The Berlin Mobil Brain/Body Imaging lab (\emph{BeMoBIL}) investigates human brain dynamics on spatial orientation tasks on subjects actual performing real body movements while navigating through and interacting with their environments. \cite{gramann2011cognition}
% Warum wird das gemacht? 
% Vielleicht - was hat das mit Human Factors  tun? 

Such a cross-disciplinary research often requires a broad range of knowledge and skills besides the actual domain of the research.
In the case of BeMoBIL, spatial cognition research needs complex setups due to following requirements:
\begin{itemize} 
	\item Building different technological setups necessary for orientation and navigation tasks beyond conventional approaches of subjects looking to a screen and doesn't move.
	\item The acquisition and statistical analytics of -physiological data corresponding to different kind of events provided by the experimental paradigm.
\end{itemize}
Further research requires the usage of virtual environments to enable a broad range of experimental setups with an higher % Validity
than conventional paradigms. 

\section*{State-of-the-Art}

% Requirements of current and future spatial cognition research 

% Divergent of evolving tool sets

\section*{Concrete research question}

Is the usage of tools made for the design and implementation of video games also useful for the creation of experimental paradigms. Especially, if they are using head-mounted virtual reality displays and real-time motion capture data? 

\section*{Methods \& Procedures}

To approach an answer to the given question a modern game engine called Unity3D will be used to implement on experimental paradigm as example. Furthermore, the quality of the approach has to be proofed.  
The usability of the components and the game engine itself will be investigated on a qualitative usability study. Also the quality of the data generated by the approach needs to be proofed, therefor what?

\subsection*{Technological Integration}
 
\begin{itemize}
\item Integrate the different hard and software components to provide easy access to all the necessary data like MotionCapturing, Sensor data of the head-mounted display. 
\item Integrate the output data through an interface already used by the data acquisition and analytics framework.
\item Provide components producing data on orientation and navigation of subjects.
\end{itemize}

\subsection*{Empirical work}



\section*{Timeline}

\bibliographystyle{plain}
\bibliography{../../literatur}

\end{document}